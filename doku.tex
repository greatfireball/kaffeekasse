\documentclass[11pt,a4paper]{IEEEtran}
\usepackage[ngerman]{babel}
\usepackage[utf8]{inputenc}

\begin{document}

\title{DIY-Projekt Kaffeekasse}
\author{\IEEEauthorblockN{Martin Hofmann}
    \IEEEauthorblockA{Universität Erlangen-Nürnberg\\
    Email: martin.hofmann@fau.de}
}

\date{6. Februar 2016}
% Just remember to make sure that the TOTAL number of authors
% is the number that will appear on the first page PLUS the
% number that will appear in the \additionalauthors section.

\maketitle
\begin{abstract}
    Abstract TODO
\end{abstract}

\section{Einführung}

Im Rahmen meiner Werkstudententätigkeit bei einer Softwarefirma, benutzte ich 
die abteilungseigene Kaffeemaschine. Um zum Ende des Monats die Kosten für
Kaffee fair nach der jeweiligen Anzahl an getrunkenen Kassen auf die Mitarbeiter
zu verteilen, wird eine Strichliste geführt.

Das manuelle Führen einer solchen Liste -- verbunden mit dem eintippen
derselben in ein Tabellenkalkulationsprogramm -- stellt den Anreiz dar, die 
Abrechnung des Kaffeekonsums zu automatisieren.

Die Möglichkeiten 

\section{Body} %\end{document}  % This is where a 'short' article might terminate
%
% The following two commands are all you need in the
% initial runs of your .tex file to
% produce the bibliography for the citations in your paper.
\bibliographystyle{abbrv}
\bibliography{sigproc}  % sigproc.bib is the name of the Bibliography in this case
% You must have a proper ".bib" file
%  and remember to run:
% latex bibtex latex latex
% to resolve all references
%
% ACM needs 'a single self-contained file'!
%
%APPENDICES are optional
%\balancecolumns
%\appendix
%Appendix A
\end{document}
